\documentclass[a4paper,17pt]{extarticle}
\usepackage[pdfpages,caption, subfig, table, xcdraw]{xcolor}
\usepackage{graphicx}
\usepackage{caption}
\usepackage{subcaption}
\usepackage{amsmath}
\usepackage{xfrac}
\usepackage{nicefrac}
\usepackage{epstopdf}
\usepackage{epstopdf}
\usepackage[utf8x]{inputenc} 
\usepackage{mathrsfs}
%\usepackage[a4paper, total={7.0in, 9.5in}]{geometry}
\usepackage[table]{xcolor}
\usepackage{tabularx}
\usepackage{makecell}
\usepackage{mathtools}
\linespread{1.5}
\usepackage{xcolor}
\usepackage{hyperref}
\usepackage{enumitem}
\usepackage[makeroom]{cancel}
\usepackage{ulem}
\usepackage{algorithm}
\usepackage{algpseudocode}
\usepackage[math]{cellspace}
\setcounter{MaxMatrixCols}{20}
\usepackage{multirow}
\usepackage{graphicx}
\usepackage{tikz}
\usetikzlibrary{trees}
\usepackage{forest}
\usetikzlibrary{positioning}

\begin{document}

\begin{titlepage}
\title{Scaling Factors Paper Notes
}

\author{
Zaid Sabri \\
NASA, GRC-LTI0 \\
Cleveland, OH  44113 \\
zaid.h.sabri@nasa.gov }

\maketitle
\end{titlepage}

\pagebreak \section*{Slide 1}
\begin{itemize}
   \item Hello everyone and welcome to my presentation where I'll be discussing "The efficiency og
      Higher-Order Implicit Solvers for Compressible Flow"
   \item My co-author on this work is Ray Hixon and we're both from the University of Toledo
\end{itemize}

\pagebreak \section*{Background - Slide 2}
\begin{itemize}
   \item Computational Fluid Dynamics (CFD) is the study of fluid mechanics that predicts fluid
      flows using digital computers.
   \item The analytical equations of fluid mechanics can be written in several forms, one of which
      is a differential form that consists of partial derivatives with respect to space and time
   \item In order to solve the equations, they are discretized in space, turning into Ordinary
      Differential equations which can be integrated to advance the flow in time
   \item The effectiveness of the time marching schemes depend on the convergence, accuracy and
      computational efficiency
   \item How much of a role are the numerical schemes effecting the solution? In order to answer
      that question we must further analyze the numerical schemes
\end{itemize}

\pagebreak \section*{Spatial Derivative - Slide 3}
\begin{itemize}
   \item Using a finite difference approximation, the continuous problem domain is ”discretized”
      so that the dependent variables are considered to exist only at discrete points
   \item Taylor series expansion is used to obtain spatial derivatives on those structured grid
      points which can then be used to derive derivative
   \item On the other hand the numerical approximation does not come for free.
   \item One issue we run into with finite differencing schemes is that we retain the exact
      analytical derivative, however it is muliplied by an error which is also a function of the 
      number of grid points used.
   \item Ploting the error for various schemes that will be shown with the Figure on the right
\end{itemize}

\pagebreak \section*{Time Marching - Slide 4}
\begin{itemize}
   \item  Moving on to the time derivative
   \item Starting with an initial state at time zero, given a boundary condition, we can 
         ”march” one time-step to a time later and use a numerical scheme to evaluate the 
         solution at that time.
   \item This process, known as time marching, is subdivided into two schemes, \textbf{explicit} and
      \textbf{implicit}.
   \item Explicit schemes can be relatively simple to implement but may require excessive 
         time steps due to a bounded stability limit.
   \item When solving viscous flow problems where the computational domain must be highly 
         clustered near the body, the size of the time step an explicit scheme can take is 
         restricted by the CFL condition. 
   \item The stability restriction can be eliminated by using an implicit time marching scheme.
\end{itemize}

\pagebreak \section*{Implicit Time Marching - Slide 5}
\begin{itemize}
   \item To derive the implicit time marching algorithm, we start with the  one-dimensional Euler
      equation
   \item We rewrite the equation in delta form and linearize the Taylor series expansion to get
         what is shown in equation (2)
   \item What this represents is the change of the flow variable {delta Q} as we move from 
         one time step to another.
   \item The RHS is a vector that represents an explicit approximation to the governing equation.
   \item Thus Whatever is multiplying deltaQ on the LHS must be a square matrix, which is known as 
         the numerics of the governing equation.
   \item This form of the equation is known as an unfactored form of the time marching scheme.
   \item The simplest way to solve this equation is by inverting the numerics and multiplying it by
         the RHS. 
   \item Due to the computational complexities faced with inverting the numerics, especially in 
         multi-dimensions, much research has focused on approximating the Numerics
\end{itemize}

\pagebreak \section*{Implicit Time Marching - Prior Work - Slide 6}
\begin{itemize}
   \item Alternating Direction Implicit known as ADI was developed in in 1970s by Beam and Warming, 
         is a practical implicit scheme that approximately factorizes the implicit operator in delta 
         form. 
   \item The idea there was the matrix was approximated as a product of two matricies, which
         significantly cut down on the computational work needed. Widely disseminated codes such as
         ARC2D/3D, CFL3D, INS3D, OVERFLOW and many more are based on the ADI scheme.
   \item The Lower-Upper Symmetric-Gauss-Seidel method (LU-SGS) developed by Yoon and Jameson is 
         another common approximation where the numerics are preconditioned to a product of three 
         matrices which provides computational efficiency. 
   \item LU-SGS is unconditionally stable in one, two, and three dimensions, unlike ADI, which is
         conditionally stable in three dimensions. 
   \item Implicit methods described above are computationally expensive not only in multidimensions
         but also when using differencing stencils with high orders of accuracy.
   \item However what if we use low order differences to approximate the numerics, while fixing
      the physics of the governing equations to higher order differences
\end{itemize}

\pagebreak \section*{Implicit Time Marching - Preconditioning - Slide 7}
\begin{itemize}
   \item Let's say an optimized RDRP differencing stencil is used in the givering equation, the
         unfactored form of the scheme is shown in equation (3)
   \item But we can precondition the numerics using a low order differnce scheme as seen in equation
         (4)
   \item The practical upshot here is that a more efficient solver is present since the number of
         diagonal on the matrix has been reduced.
   \item Typically approximating a scheme yields additional error terms which can lead to
         instability, however this can be counteracted by adding implicit dissipation to stabilize
         the scheme. A computer code known as FDL3DI uses this approach
\end{itemize}

\pagebreak \section*{Stability - Slide 8}
\begin{itemize}
   \item The approach used in this investigation is similar to that used by FDL3DI; but instead of
      adding dissipation to the numerics, stability is obtained here by scaling errors from the
      numerics to overcome errors from the physics of the governing equations.
   \item So the equation is approxumated one last time as seen in equation (5), where a sclaing
         factor, sigma, multiplies the spatial difference on the LHS. 
   \item Hence the scaling factor is physically insuring the numerical errors from the second-order
         difference can overcome the numerical errors from the RHS differencing stencil
   \item Values of the scaling factors for both a periodic and bounded flows are shown in the table 
         below.
\end{itemize}

\pagebreak \section*{Numerical Results for Benchmark Problems - Slide 9}
\begin{itemize}
   \item To validate our conclusions, we’ll be looking at Cat- egory 1 problem 1 from the third
      computational aeroacoustic workshop.
   \item Problem 1 models the upstream propagation of sound through a nozzle with near sonic 
         conditions.
\end{itemize}

\pagebreak \section*{Category 1 Problem 1, Steady State Results - Slide 10}
\begin{itemize}
   \item The steady mean-flow results are analyzed at first with the following conditions
   \item Here the preconditioned matrix is tested tested against 6 different differencing stencils,
      we have second fourth and sixth order scheme. DRP and RDPR are optimized scheme and a
      Prefactored Fourth Order Compact Scheme is also present
   \item Figure on the left evaluates the Domain vs Mean pressure. Notice that the steady results
      for all scheme agree very well with the exact solution. More importantly the preconditioned
      scheme are stable for steady-state results
   \item Figure on the right evalues the number iterations vs Residual of the fluxes. 
   \item As seen on the plot, a second order difference will converge quicker than a higher order 
         difference
   \item That is because the closer our RHS is to the LHS, the less of a factorization error we get
\end{itemize}

\pagebreak \section*{Category 1 Problem 1, Unsteady Results - Slide 11}
\begin{itemize}
   \item Once the steady mean-flow results are obtained, the upstream propagating perturbation
         starts at the boundary, and the unsteady results are studied.
   \item Here, a small amplitude acoustic wave is generated way downstream and propagate upstream
      through the narrow passage of the nozzle throat
   \item The figures shown are analyzing the periodic steady-state solutions. What I want to
         highlight out of this figure is how the higher order schemes results agree with the exact
         solution, but the low order scheme does not. The CFL used in this problem is shown in the
         Table above, we're obtaining CFL as high as 22 for a prefactored compact scheme while fully
         converging down to machine precision on each implicit solve. 
\end{itemize}

\end{document}

